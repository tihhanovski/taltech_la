\documentclass[10pt, a4paper]{article}
\usepackage[margin=1.0in]{geometry}
\usepackage[utf8]{inputenc}

\addtolength{\oddsidemargin}{0.18in}
\addtolength{\evensidemargin}{-0.5in}
\addtolength{\textwidth}{0.32in}

%TODO: why it breaks latex?
%\usepackage{fontspec}
%\setmainfont{Verdana}

\usepackage{gensymb}
\usepackage{fancyhdr}
\setlength{\headheight}{14pt}

\usepackage{amsmath}
\usepackage{cancel}

\pagestyle{fancy}
\fancyhf{}
\rhead{Ilja Tihhanovski, KFS2}
\lhead{LA kodutöö 3}
\rfoot{\thepage}

\title{Lineaaralgebra kodutöö 2}
\author{Ilja Tihhanovski, KFS 2}
\date{September 2020}

\linespread{1.5}    

\begin{document}

    \section{Ülesanne 1}
        Kolmnurga tipud: A(1; 4; 1), B(4; -2; 1), C(-4; -1; 1).
        \begin{enumerate}
            \item %kolmnurga külgede pikkused
                \begin{align}
                    \nonumber
                    \vec{AB} &= (4 - 1; -2 - 4; 1 - 1) = (3; -6; 0),\\ \nonumber
                    \vec{BC} &= (-4 - 4; -1 - (-2); 1 - 1) = (-8; 1; 0),\\ \nonumber
                    \vec{CA} &= (1 - (-4); 4 - (-1); 1 - 1) = (5; 5; 0), \\ \nonumber
                    \lvert\vec{AB}\rvert &= \sqrt{3^2 + (-6)^2 + 0^2} = \sqrt{9 + 36} = \sqrt{45} = \underline{3\sqrt{5}}, \\ \nonumber
                    \lvert\vec{BC}\rvert &= \sqrt{(-8)^2 + 1^2 + 0^2} = \sqrt{64 + 1} = \underline{\sqrt{65}}, \\ \nonumber
                    \lvert\vec{CA}\rvert &= \sqrt{5^2 + 5^2 + 0^2} = \sqrt{25 + 25} = \sqrt{50} = \underline{5\sqrt{2}};
                \end{align}
                \textbf{Vastus: } Kolmnurga külgede pikkused on $3\sqrt{5}$, $\sqrt{65}$ ja $5\sqrt{2}$.
                
            \item %kolmnurga sisenurgad
                Kasutame eelmises punktis leitud vektorite koordinaate ja pikkusi.\\ Arvestame sellega, et $\lvert \vec{AB} \rvert = \lvert \vec{BA} \rvert$ ja $\vec{AB} = - \vec{BA}$:
                \begin{align}
                    \nonumber
                    \angle A &= \arccos \frac{\langle \vec{AB}, \vec{AC} \rangle}{\lvert \vec{AB} \rvert \cdot \lvert \vec{AC} \rvert} 
                        = \arccos \frac{\langle \vec{AB}, -\vec{CA} \rangle}{\lvert \vec{AB} \rvert \cdot \lvert \vec{CA} \rvert} 
                        = \arccos \frac{\langle (3; -6; 0), (-5; -5; 0) \rangle}{\sqrt{45} \cdot \sqrt{50}} = \\ \nonumber
                        &= \arccos \frac{3 \cdot (-5) + (-6) \cdot (-5) + 0}{\sqrt{2250}} =
                        \arccos \frac{-15 + 30}{\sqrt{2250}} =\\ \nonumber
                        &=\arccos \frac{15}{\sqrt{2250}} \approx \underline{71\degree 33' 54''},\\ 
                        \nonumber
                    \angle B &= \arccos \frac{\langle \vec{BA}, \vec{BC} \rangle}{\lvert \vec{BA} \rvert \cdot \lvert \vec{BC} \rvert} 
                        = \arccos \frac{\langle -\vec{AB}, \vec{BC} \rangle}{\lvert \vec{AB} \rvert \cdot \lvert \vec{BC} \rvert} 
                        = \arccos \frac{\langle (-3; 6; 0), (-8; 1; 0) \rangle}{\sqrt{45} \cdot \sqrt{65}} = \\ \nonumber
                        &= \arccos \frac{3 \cdot 8 + 6 \cdot 1 + 0}{\sqrt{2925}} =
                        \arccos \frac{24 + 6}{\sqrt{2925}} =\\ \nonumber
                        &=\arccos \frac{30}{\sqrt{2925}} \approx \underline{56\degree 18' 36''},\\ 
                        \nonumber
                    \angle C &= \arccos \frac{\langle \vec{CA}, \vec{CB} \rangle}{\lvert \vec{CA} \rvert \cdot \lvert \vec{CB} \rvert} 
                        = \arccos \frac{\langle \vec{CA}, -\vec{BC} \rangle}{\lvert \vec{CA} \rvert \cdot \lvert \vec{BC} \rvert} 
                        = \arccos \frac{\langle (5; 5; 0), (8; -1; 0) \rangle}{\sqrt{50} \cdot \sqrt{65}} = \\ \nonumber
                        &= \arccos \frac{5 \cdot 8 + 5 \cdot (-1) + 0}{\sqrt{3250}} =
                        \arccos \frac{40 - 5}{\sqrt{3250}} =\\ \nonumber
                        &=\arccos \frac{35}{\sqrt{3250}} \approx \underline{52\degree 7' 30''};
                \end{align}
                Kontrollimiseks võime liita saadud tulemused - kolmnurga nurkade summa peab võrduma 180\degree:
                \begin{align}
                    \nonumber
                    \angle A + \angle B + \angle C &= 71\degree 33' 54'' + 56\degree 18' 36'' + 52\degree 7' 30'' = 180\degree;
                \end{align}
                \textbf{Vastus: } Nurgad A, B ja C on vastavalt $71\degree 33' 54''$, $56\degree 18' 36''$ ja $52\degree 7' 30''$.
                
            \item
                $S_\triangle = \frac{1}{2}S_{r}$, kus rööpküliku pindala $S_{r} = \lvert \vec{AB} \times \vec{BC} \rvert$.\\
                Võime võtta kaks suvalist vektorit kolmest. Kuna võtame vektorkorrutise absoluutväärtuse, ei ole vahet, kas kasutame $\vec{AB}$ või $\vec{BA}$. $\vec{AB} \times \vec{BC} = -(\vec{BA} \times \vec{BC})$.
                \begin{align}
                    \nonumber
                    \vec{AB} \times \vec{BC} &= (3; 6; 0) \times (-8; 1; 0) = \\ \nonumber
                    &= \begin{pmatrix}
                        \begin{vmatrix}
                        -6 & 0\\
                        1 & 0
                        \end{vmatrix};
                        -\begin{vmatrix}
                        3 & 0\\
                        -8 & 0
                        \end{vmatrix};
                        \begin{vmatrix}
                        3 & -6\\
                        -8 & 1
                        \end{vmatrix}
                    \end{pmatrix} =  \\ \nonumber
                    &= (0; 0; 3 \cdot 1 - (-6) \cdot (-8)) = (0; 0; -45),\\ \nonumber
                    S_\triangle &= \frac{1}{2}\lvert \vec{AB} \times \vec{BC} \rvert = \frac{\sqrt{(-45)^2}}{2} = \frac{45}{2} = \underline{22,5};
                \end{align}
                \textbf{Vastus: }Kolmnurga pindala on 22,5 ruutühikut.
        \end{enumerate}
        
    \section{Ülesanne 2}
        $\vec{a} = (4; -2; 5), \vec{b} = (-1; 0; 3), \vec{c} = (1; 2; -4)$:
        \begin{enumerate}
            \item Vektorkorrutis $\vec{a} \times \vec{b}$:
                \begin{align}
                    \nonumber
                    \vec{a} \times \vec{b} &= 
                    \begin{pmatrix}
                        \begin{vmatrix}
                            a_2 & a_3 \\
                            b_2 & b_3
                        \end{vmatrix};
                        -\begin{vmatrix}
                            a_1 & a_3 \\
                            b_1 & b_3
                        \end{vmatrix};
                        \begin{vmatrix}
                            a_1 & a_2 \\
                            b_1 & b_2
                        \end{vmatrix}
                    \end{pmatrix}
                    = \begin{pmatrix}
                        \begin{vmatrix}
                            -2 & 5 \\
                            0 & 3
                        \end{vmatrix};
                        -\begin{vmatrix}
                            4 & 5 \\
                            -1 & 3
                        \end{vmatrix};
                        \begin{vmatrix}
                            4 & -2 \\
                            -1 & 0
                        \end{vmatrix}
                    \end{pmatrix} = \\ \nonumber
                    &=(-2 \cdot 3 - 5 \cdot 0; -(4 \cdot 3 - 5 \cdot (-1)); 4 \cdot 0 - (-2) \cdot (-1)) = \\ \nonumber
                    &= (-6 - 0; -(12 - (-5)); 0 - 2) = \underline{(-6; -17; -2)},
                \end{align}
            \item Segakorrutis $(\vec{a} \vec{b} \vec{c})$:
                \begin{align}
                    \nonumber
                    (\vec{a} \vec{b} \vec{c}) &=
                    \begin{vmatrix}
                        4 & -2 & 5 \\
                        -1 & 0 & 3 \\
                        1 & 2 & -4
                    \end{vmatrix} = \\ \nonumber
                    &= 4 \cdot 0 \cdot (-4) + (-2) \cdot 3 \cdot 1 + 5 \cdot (-1) \cdot 2 - (5 \cdot 0 \cdot 1 + (-2) \cdot (-1) \cdot (-4) + 4 \cdot 3 \cdot 2) = \\ \nonumber
                    &= 0 - 6 - 10 - 0 + 8 - 24 = \underline{-32},
                \end{align}
                Saame segakorrutist leida ka teisiti, punktis 1 leitud vektorkorrutise abil:
                \begin{align}
                    \nonumber
                    (\vec{a} \vec{b} \vec{c}) &= \langle (\vec{a} \times \vec{b}) \cdot \vec{c} \rangle = -6 \cdot 1 + (-17) \cdot 2 + (-2) \cdot (-4) = -6 - 34 + 8 = \underline{-32},
                \end{align}
            \item Rööpküliku pindala leidmiseks vajaliku $\vec{a}$ ja $\vec{b}$ vektorkorrutise leidsime punktis 1.
                \begin{align}
                    \nonumber
                    S_{\vec{a}\vec{b}} &= \lvert \vec{a} \times \vec{b} \rvert = \sqrt{(-6)^2 + (-17)^2 + (-2)^2} = \sqrt{36 + 289 + 4} = \underline{\sqrt{329}},
                \end{align}
            \item Rööptahuka ruumala leidmiseks vajaliku segakorrutise leidsime punktis 2.
                \begin{align}
                    \nonumber
                    V_{\vec{a}\vec{b}\vec{c}} &= \lvert (\vec{a} \vec{b} \vec{c}) \rvert = \lvert -32 \rvert = \underline{32};
                \end{align}
            \end{enumerate}
        \textbf{Vastus: } Vektorkorrutis $\vec{a} \times \vec{b} = (-6; -17; -2)$ , Segakorrutis $(\vec{a} \vec{b} \vec{c}) = -32$, rööpküliku pindala on $\sqrt{329}$ ruutühikut, rööptahuka ruumala on 32 kuupühikut.
    
    \begin{verbatim}
        https://github.com/tihhanovski/taltech_la/blob/master/homework3/homework3.tex
    \end{verbatim}
\end{document}
