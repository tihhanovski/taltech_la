\documentclass[10pt, a4paper]{article}
\usepackage[margin=1.0in]{geometry}

\addtolength{\oddsidemargin}{0.18in}
\addtolength{\evensidemargin}{-0.5in}
\addtolength{\textwidth}{0.32in}

%\addtolength{\topmargin}{-.875in}
%\addtolength{\textheight}{1.75in}
	
\usepackage{gensymb}
\usepackage{fontspec}
\setmainfont{Verdana}
\usepackage{fancyhdr}
\setlength{\headheight}{14pt}

%\usepackage{graphicx}
%\usepackage{caption}
%\captionsetup[figure]{font=small}
\usepackage{amsmath}
\usepackage{cancel}
%\usepackage[]{siunitx}
%\usepackage[style=numeric,sorting=none]{biblatex}
%\addbibresource{references.bib}
\pagestyle{fancy}
\fancyhf{}
\rhead{Ilja Tihhanovski, KFS2} % same for your university
\lhead{LA kodutöö 2} % change this to your actual name
\rfoot{\thepage}

\title{Lineaaralgebra kodutöö 2}
\author{Ilja Tihhanovski, KFS 2}
\date{September 2020}

\linespread{1.5}    
%siin soovitatakse 1.25 kui ms wordi 1.5-le vastavat. https://latex.org/forum/viewtopic.php?t=28685
\begin{document}

% title
%\maketitle


\section{Ülesanne 1}
Antud on punktid $A(2; 0; 4)$, $B(−1; 2; −3)$, $C(4; −3; −2)$ ja $D(−2; −1; −5)$.
\begin{enumerate}

%1. vektorid AB ja DC
\item 
\begin{align}
    \vec{AB} &= (-1 - 2; 2 - 0; -3 - 4) = \underline{(-3; 2; -7)},\\
    \vec{DC} &= (4 - (-2); -3 - (-1); -2 - (-5)) = \underline{(6; -2; 3)};
\end{align}
\textbf{Vastus}: $\vec{AB} = (-3; 2; -7)$, $\vec{DC} = (6; -2; 3)$.

%2. vektorite BA~ ja DC~ pikkus
\item
Kasutame varem leitud (1, 2) vektorite $\vec{AB}$ ja $\vec{DC}$ koordinaate ning asjaolu, et $\vec{BA} = -\vec{AB})$:
\begin{align}
    \nonumber
    \lvert\vec{BA}\rvert &= \lvert-(-3; 2; -7)\rvert = \lvert(3; -2; 7)\rvert = \sqrt{3^2 + (-2)^2 + 7^2} = \sqrt{9 + 4 + 49} = \underline{\sqrt{62}},\\
    \nonumber
    \lvert\vec{DC}\rvert &= \lvert(-6; -2; 3)\rvert = \sqrt{(-6)^2 + (-2)^2 + (3)^2} = \sqrt{36 + 4 + 9} =  \sqrt{49} = \underline{7};
\end{align}
\textbf{Vastus}: $\lvert\vec{BA}\rvert = \sqrt{62}$, $\lvert\vec{DC}\rvert = 7$.

%3. lõigu BD keskpunkt M
\item
Lõigu $BD$ keskpunkti $M$ koordinaatide leidmiseks üks viis on liita vektor $\vec{BM}$ punkti $B$ koordinaatidele (või vektorile $\vec{OB}$). Kuna punkt $M$ on lõogu $BD$ keskpunkt, siis $\vec{BM} = \frac{1}{2}\vec{MD}$. Teisisõnu $\vec{OM} = \vec{OB} + \frac{1}{2}\vec{BD}$.\\
(Siin ja edasi $O$ all mõtleme koordinaatide alguspunkti $(0; 0; 0)$ ja $\vec{OX}$ all punkti $X$ kohavektorit).
\begin{align}
    \nonumber
    \vec{OM} &= (-1; 2; -3) + \frac{1}{2}(-2-(-1); -1-2; -5-(-3)) = \\
    \nonumber
    &=(-1; 2; -3) + \frac{1}{2}(-1; -3; -2) = \\
    \nonumber
    &=(-1; 2; -3) + (-\frac{1}{2}; -\frac{3}{2}; -1) = \\
    \nonumber
    &=\underline{(-1\frac{1}{2}; \frac{1}{2}; -4)}
\end{align}
\textbf{Vastus}: $M(-1,5; 0,5; -4)$.

Teine võimalus lõigu keskpunkti leidmiseks, mis tuleb meelde, on valem, kus tuli poolitada lõigu otste koordinaatide summa: $X_{keskmine} = \frac{X_1 + X_2}{2}$.
 \begin{align}
    \nonumber
    \begin{array}{l}
     x_M = \frac{x_B + x_D}{2} = \frac{-1 -2}{2} = -\frac{3}{2} = -1,5\\
     y_M = \frac{y_B + y_D}{2} = \frac{2 - 1}{2} = \frac{1}{2} = 0,5\\
     z_M = \frac{z_B + z_D}{2} = \frac{-3 - 5}{2} = -\frac{8}{2} = -4\\
     \end{array} \Bigg| &\Rightarrow M(-1,5; 0,5; -4)
\end{align}
Tegelikult teine valem järeldub esimesest:
\begin{align}
    \nonumber
    \vec{OB} + \vec{BM} &= \vec{OB} + \frac{\vec{BD}}{2} =\\
    \nonumber
    &= \vec{OB} + \frac{\vec{OD} - \vec{OB}}{2} = \\ \nonumber
    &= \vec{OB} + \frac{\vec{OD}}{2} - \frac{\vec{OB}}{2} \\ \nonumber
    &= \frac{\vec{OB}}{2} + \frac{\vec{OD}}{2} =\\ \nonumber
    &= \frac{\vec{OB} + \vec{OD}}{2}
\end{align}

%4. vektorite summa CA + BD
\item
\begin{align}
    \nonumber
    \vec{CA} + \vec{BD} &= (2 - 4; 0 - (-3); 4 - (-2)) + (-2 - (-1); -1 - 2; -5 - (-3)) =\\ \nonumber
    &= (-2; 3; 6) + (-1; -3; -2) = (-2-1; 3-3; 6-2) = \underline{(-3; 0; 4)}
\end{align}
\textbf{Vastus}: $\vec{CA} + \vec{BD} = (-3; 0; 4)$.

%5) vektorite vahe DC~ − AC~ ;
\item
\begin{align}
    \nonumber
    \vec{DC} - \vec{AC} &= (\vec{OC} - \vec{OD}) - (\vec{OC} - \vec{OA}) = \\ \nonumber
    &= \cancel{\vec{OC}} - \vec{OD} - \cancel{\vec{OC}} + \vec{OA} = \\ \nonumber
    &= \vec{OA} - \vec{OD} = \vec{DA} = \\  \nonumber
    &= (2 - (-2); 0 - (-1); 4 - (-5)) = \underline{(4; 1; 9)}
\end{align}
\textbf{Vastus}: $\vec{DC} - \vec{AC} = (4; 1; 9)$.
 
%6) vektori ja skalaari korrutis λBA~ , kui λ = −5;
\item
    Kasutame punktis 1 leitud $\vec{AB}$:
    \begin{align}
        \nonumber
        \lambda\vec{BA} &= -5(-\vec{AB}) = 5\vec{AB} = 5(-3; 2; -7) = 
        \underline{(-15; 10; -35)}
    \end{align}
    \textbf{Vastus}: $\lambda\vec{BA} = (-15; 10; -35)$.

%7) vektoritele BD~ ja AC~ vastavad ühikvektorid;
\item
    $\vec{BD}_0$:
    \begin{align}
        \nonumber
        \vec{BD} &= (-2 - (-1); -1 - 2; -5 - (-3)) = (-1; -3; -2),\\ 
        \nonumber
        \lvert BD \rvert &= \sqrt{(-1)^2 + (-3)^2 + (-2)^2} = \sqrt{1 + 9 + 4} + \sqrt{14}, \\ 
        \nonumber
        \vec{BD}_0 &= \frac{\vec{BD}}{\lvert BD \rvert} = \frac{(-1; -3; -2)}{\sqrt{14}} = \underline{(-\frac{1}{\sqrt{14}}; -\frac{3}{\sqrt{14}}; -\frac{2}{\sqrt{14}})};
    \end{align}    
    $\vec{AC}_0$:    
    \begin{align}
        \nonumber
        \vec{AC} &= (4 - 2; -3 - 0; -2 - 4) = (2; -3; -6),\\
        \nonumber
        \lvert AC \rvert &= \sqrt{2^2 + (-3)^2 + (-6)^2} = \sqrt{4 + 9 + 36} + \sqrt{49} = 7,\\ 
        \nonumber
        \vec{AC}_0 &= \frac{\vec{AC}}{\lvert AC \rvert} = \frac{(2; -3; -6)}{7} = \underline{(\frac{2}{7}; -\frac{3}{7}; -\frac{6}{7})};
    \end{align}
    \textbf{Vastus}: $\vec{BD}_0 = (-\frac{1}{\sqrt{14}}; -\frac{3}{\sqrt{14}}; -\frac{2}{\sqrt{14}}), \vec{AC}_0 = (\frac{2}{7}; -\frac{3}{7}; -\frac{6}{7})$.

%8) nurk punktile A vastava kohavektori OA~ ja x-telje vahel;
\item
    Lakoonilisuse huvides nimetame otsitav nurk vektori $\vec{OA}$ ja $x-$telje vahel $\alpha$-ks.
    \begin{align}
        \nonumber
        \vec{OA} &= (2; 0; 4), \\
        \nonumber
        x_A &= 2, \\
        \nonumber
        \lvert OA \lvert &= \sqrt{2^2 + 0^2 + 4^2} = \sqrt{4 + 16} = \sqrt{20}, \\
        \nonumber
        \cos \alpha &= \frac{x_A}{\lvert OA \lvert} = \frac{2}{\sqrt{20}} = \frac{1}{\sqrt{5}},\\
        \nonumber
        \alpha &= \arccos \frac{1}{\sqrt{5}} \approx \underline{63,4\degree};
    \end{align}
    \textbf{Vastus}: $\alpha \approx 63,4\degree$.

%9) nurk punktile D vastava kohavektori OD~ ja y-telje vahel;
\item
    $\alpha$ on otsitav nurk:
    \begin{align}
        \nonumber
        \vec{OD} &= (-2; -1; -5), \\
        \nonumber
        y_D &= -1, \\
        \nonumber
        \lvert OD \lvert &= \sqrt{(-2)^2 + (-1)^2 + (-5)^2} = \sqrt{4 + 1 + 25} = \sqrt{30}, \\
        \nonumber
        \cos \alpha &= \frac{y_D}{\lvert OD \lvert} = \frac{-1}{\sqrt{30}},\\
        \nonumber
        \alpha &= \arccos -\frac{1}{\sqrt{30}} \approx \underline{100,5\degree};
    \end{align}
    \textbf{Vastus}: $\alpha \approx 100,5\degree$.\\
    Oli oodata, et $\alpha > 90\degree$, sest mõõdame seda positiivse $y$-telje suunast ning $y_D$ on negatiivne ($90\degree < \alpha < 180\degree \Rightarrow \cos \alpha < 0 $).

%10) nurk punktile C vastava kohavektori OC~ ja z-telje vahel;
\item
$\alpha$ on otsitav nurk:
\begin{align}
    \nonumber
    \vec{OC} &= (4; -3; -2), \\
    \nonumber
    z_C &= -2, \\
    \nonumber
    \lvert OC \lvert &= \sqrt{4^2 + (-3)^2 + (-2)^2} = \sqrt{16 + 9 + 4} = \sqrt{29}, \\
    \nonumber
    \cos \alpha &= \frac{z_D}{\lvert OC \lvert} = \frac{-2}{\sqrt{29}},\\
    \nonumber
    \alpha &= \arccos -\frac{2}{\sqrt{29}} \approx \underline{111,8\degree};
\end{align}
\textbf{Vastus}: $\alpha \approx 111,8\degree$.\\

\end{enumerate}

\end{document}
